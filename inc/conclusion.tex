\chapter{Conclusion and Further Work}
\label{chap:conclusion}
\section{Conclusion}
In this project thesis, the aim was to do the necessary preparations for a master thesis that will be written and delivered in the spring of 2019. The project thesis consists of a literature review, theory about the system, description of the global and local model as well as a brief overview of the numeric background for SIMA RIFLEX and BFLEX. This project thesis has been a great preparation for the master thesis. Through the literature review, the development in recent years of wind turbines and offshore wind has been investigated, and it is clear that there has been a strong positive development with increasing size, efficiency and capacity for offshore wind turbines. As the offshore wind installations move further from shore and into deeper waters, accurate estimation of the lifetime of the dynamic power cables is crucial. The conclusion of the literature review is that this topic is very interesting and important, and definitely worth continuing within the master thesis. \\\\ To prepare for the analyses that will take place in the next semester, a local and a global model was created. Both programs use co-rotated beams as element formulation, and they both use a method combining the Euler-Cauchy Method and the Newton Raphson Method for the static non-linear analyses. For the dynamic analysis, SIMA RIFLEX uses Newark's $\beta$ while BFLEX uses the HHT-$\alpha$ method. \\\\ The global model consists of the total cable with lazy wave configuration modeled with beam elements and attached to the vessel. Transfer functions from Dr. Techn. Olav Olsen are used to model the response of the vessel. The global model was tested in the most severe sea state, it seems that the model performs well and that it is ready to be used in the master thesis.  \\\\The local model only includes the upper part of the cable. It was modeled with several different BFLEX elements to describe the different layers in the cross section and the contact between them as accurately as possible. The local model also includes a bend stiffener to add local stiffness to the model. The local model was tested by applying tension, initial strain and dynamic bending to 5 degrees and - 5 degrees. From these tests, it seems that the bend stiffener is not having the desired effect on the cable, as it might be too short. This means that the local model will need some adjustments before it is ready to be used in analyses in the master thesis.  \\\\ As a conclusion, it can be said that this project thesis has been a very good preparation for the master thesis. The author has gained great insight into several topics such as power cables, offshore wind, fatigue and more, as well as learning two new computer programs. In addition, this has been a good exercise in project work. Making decisions, modeling, making assumptions and simplifications and writing reports are all parts of being an engineer. It will be very interesting to continue with this project, and the author is eager to embark on the master thesis next semester.

\section{Further Work}
The further work on this project will be to complete the master thesis. In the master thesis, the actual analyses will be executed to estimate the lifetime of the power cable. To do this in the most efficient way, a script will need to be made that will run all the sea states in the global analysis and give out the time series of the angel and the tension. An appropriate rain flow counting algorithm will have to be selected and implemented in order to count the tension and angle cycles. The results from the rain flow counting will be the input for the local analyses, and this will be used to calculate the stress variation in each individual wire so that the lifetime can be estimated based on SN data. \\\\Before the modeling can begin, the dimensions of the bend stiffener will have to be adjusted, as it does not fulfill its purpose at this point. Another thing that could be looked into further are the friction models describing the contact between the layers, and it will also be interesting to investigate the effect of the new element HCONT454. In terms of the global model, the configuration of the cable could possibly be optimized further, and the effect of wind could be made more complex. 

