\chapter*{Abstract}
Man has utilized wind for thousands of years. In the beginning, it was mostly for propulsion for ships in the form of sails. Later primitive windmills were used to grind cereal and pump water. The first ever wind turbine was created in 1888, and in the last century, the wind turbine technology has gone through major development increasing in size, capacity, and efficiency. In the most recent years, wind turbines have been moved offshore to reduce noise and visibility and to utilize the higher wind speeds offshore. As offshore wind energy installations are moving into deeper waters, and further away from shore, floating wind installations are necessary. The dynamic power cable that will deliver the electricity from the installation is an important component for this technology. There has been a lot of research on the fatigue life on dynamic slender structures in the oil and gas industry, but very little in dynamic power cables applied in offshore wind. \newline
\newline
 To explore this further, the floating wind turbine OO-Star was chosen as a case study. This is a design done by Dr. Techn. Olav Olsen, and is participating in the project Lifes50+. To investigate the fatigue life of the dynamic power cable, two different models were created. A local and a global. SIMA RIFLEX was used for the global model, where the whole cable is modeled and is attached to a point that will have the movements of the OO-Star due to the transfer functions provided by Dr. Techn. Olav Olsen. The configuration of the cables was based on a max curvature that could not be exceeded in the static or dynamic analysis for neither position. The local model was created in BFLEX, and only the upper part of the cable was modeled as it was assumed that this is were the largest fatigue damage will be located. This model consists of the cable and a bend stiffener to add local stiffness to the cable.\newline 
\newline
Also included in the project thesis are all the necessary theory necessary to execute the procedure described above. This includes theory about wind turbines, power cables, fatigue and the numeric theory behind the software used to create the two models. This has been a very good preparation for the master thesis
